\subsection{Optical Communications System Advantages}

There are many advantages to using optical communications systems over
traditional communications systems using, for example, copper wire. Optical
communications systems allow for greater resistance to interference and signal
attenuation, which can cause problems in traditional systems.

\subsubsection{Interference}

Interference can be present in traditional data transmissions through a
phenomenon known as ``Crosstalk'' which occurs when signals within different
channels of a system have an adverse, interfering effect on one
another\cite{wikipediacross_2019}. While this is an issue within the traditional
data communications systems, limiting the amount of lines which can be run in
close proximity, it poses no such issues within optical communication systems
(provided there is adequate cladding in place\cite{mickelson_2003}).

\par Optical data transmission is relatively resistant to interference, both in
terms of Electromagnetic, and Radio-Frequency interference, thus making them
more suitable than metal based, traditional transmission systems in situations
where there is high likelihood of these types of interference\cite{alwayn_2004}.

\subsubsection{Attenuation}

Attenuation in optical systems is much less of an issue than in its metal based
counterpart. This is due to attenuation being mainly caused by absorption and
scattering within the transmission medium\cite{alwayn_2004}. Within glass cable,
this attenuation is extremely low, however, with plastic core cable, it can be
higher, due to impurities within the material\cite{alwayn_2004}.

\par Attenuation within traditional systems is higher due to the ``skin effect''
which increases attenuation at higher frequencies\cite{hayt_buck_2019}. Due to
this effect, systems utilising copper wire require repeaters at approximately
2-5km intervals, compared to a distance of approximately 50km in fiber optic
cabling\cite{mickelson_2003}\cite{fiber_vs_wire}.
